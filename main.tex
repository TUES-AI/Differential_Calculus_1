\documentclass[12pt]{article}

% ---------- Пакети ----------
\usepackage[bulgarian]{babel}
\usepackage[T1]{fontenc}
\usepackage[utf8]{inputenc}
\usepackage{amsmath, amssymb}
\usepackage{geometry}
\usepackage{tikz}
\usepackage{pgfplots}
\pgfplotsset{compat=1.18}

\geometry{margin=2.5cm}

\title{\textbf{Функции и интуиция за граници}\\
Подготовка за диференциално смятане и AI}
\author{}
\date{}

\begin{document}
\maketitle

% =================================================
\section*{0. Цел на лекцията}

Днес:
\begin{itemize}
  \item изграждаме интуиция за функции и графики
  \item учим наклон, растене/спадане, четност/нечетност
  \item извеждаме идеята за граница чрез площ под \(y=x^2\) на \([0,1]\)
\end{itemize}

% =================================================
\section*{1. Какво е функция }

Функция е правило, което на всеки допустим вход \(x\) съпоставя точно една стойност \(f(x)\).

\begin{quote}
Един и същ вход \(\Rightarrow\) един и същ изход.
Ако при един и същ вход понякога получаваме различни резултати — това \textbf{не} е функция.
\end{quote}

Записът \(f(x)\) означава:
\begin{quote}
„Резултатът, който функцията дава при вход \(x\).“
\end{quote}

\subsection*{Област на дефиниция}
Не всички \(x\) са позволени. Например:
\[
f(x)=\sqrt{x}\quad \Rightarrow \quad x\ge 0
\]
(ако работим в реалните числа).

% =================================================
\section*{2. Графика на функция}

Графиката на функцията е начин да видим всички входове и изходи наведнъж.

\begin{itemize}
  \item хоризонтална ос (абсциса) — вход \(x\)
  \item вертикална ос (ордината) — изход \(f(x)\)
\end{itemize}

Всяка точка \((x,f(x))\) означава:
\begin{quote}
„При този вход получаваме този изход.“
\end{quote}

\subsection*{Вертикален тест}
Ако една вертикална права пресича „графиката“ повече от веднъж, това не е графика на функция
(защото един вход би имал два изхода).

% =================================================
\section*{3. Линейни функции и наклон}

Линейна функция:
\[
f(x)=ax+b
\]
\begin{itemize}
  \item \(b=f(0)\) — пресечната точка с оста \(y\)
  \item \(a\) — наклон
\end{itemize}

\subsection*{Интуиция за наклона}
Наклонът отговаря на въпроса:
\begin{quote}
Колко се променя \(f(x)\), когато \(x\) се увеличи с 1?
\end{quote}

Пример: \(f(x)=2x+1\).
\[
f(1)=3,\quad f(2)=5
\]
Когато \(x\) нараства с 1, стойността на функцията нараства с 2. Значи наклонът е 2.

\subsection*{Расте/спада}
\begin{itemize}
  \item ако \(a>0\) — функцията расте (графиката се „качва“ надясно)
  \item ако \(a<0\) — функцията спада (графиката „слиза“ надясно)
  \item ако \(a=0\) — функцията е константна
\end{itemize}

\subsection*{Графика (линейни примери)}
\begin{center}
\begin{tikzpicture}
\begin{axis}[
    axis lines=middle, grid=both,
    xmin=-3,xmax=3,ymin=-2,ymax=5,
    xlabel={$x$}, ylabel={$y$},
    legend style={at={(0.02,0.98)},anchor=north west},
    width=11cm, height=7cm
]
\addplot[thick, blue, domain=-3:3] {2*x+1};
\addlegendentry{$y=2x+1$ (расте)}
\addplot[thick, red, domain=-3:3] {-x+2};
\addlegendentry{$y=-x+2$ (спада)}
\addplot[thick, green!60!black, domain=-3:3] {1};
\addlegendentry{$y=1$ (константа)}
\end{axis}
\end{tikzpicture}
\end{center}

\subsection*{Микро-задачи}
\begin{enumerate}
  \item Какъв е наклонът на \(y=5x-7\)? Расте ли или спада?
  \item Намери \(b\) (пресечната точка с \(y\)-оста) за \(y=-3x+4\).
  \item Ако графиката е права линия, която минава през \((0,2)\) и \((1,5)\), намерете линейната функция.
\end{enumerate}

\subsection*{Отговори } 
\begin{itemize}
  \item \(y=5x-7\): наклон \(a=5\), \textbf{расте}.
  \item \(y=-3x+4\): \(b=4\).
  \item През \((0,2)\) и \((1,5)\): наклон \(a=\dfrac{5-2}{1-0}=3\), \(b=2\), \(y=3x+2\).
\end{itemize}

% =================================================
\section*{4. Квадратна функция и „форма“}

Квадратна функция:
\[
f(x)=x^2
\]

Интуиция:
\begin{itemize}
  \item винаги е \(\ge 0\)
  \item около 0 расте бавно, далеч от 0 расте бързо
\end{itemize}

\subsection*{Графика}
\begin{center}
\begin{tikzpicture}
\begin{axis}[
    axis lines=middle, grid=both,
    xmin=-3,xmax=3,ymin=-1,ymax=10,
    xlabel={$x$}, ylabel={$y$},
    width=11cm, height=7cm
]
\addplot[thick, purple, domain=-3:3] {x^2};
\end{axis}
\end{tikzpicture}
\end{center}

\subsection*{Още една графика: параболи с различна „ширина“}
\begin{center}
\begin{tikzpicture}
\begin{axis}[
    axis lines=middle, grid=both,
    xmin=-3,xmax=3,ymin=-1,ymax=10,
    xlabel={$x$}, ylabel={$y$},
    legend style={at={(0.02,0.98)},anchor=north west},
    width=11cm, height=7cm
]
\addplot[thick, purple, domain=-3:3] {x^2};
\addlegendentry{$y=x^2$}
\addplot[thick, orange, domain=-3:3] {2*x^2};
\addlegendentry{$y=2x^2$ (по-тясна)}
\addplot[thick, teal, domain=-3:3] {0.5*x^2};
\addlegendentry{$y=\frac12 x^2$ (по-широка)}
\end{axis}
\end{tikzpicture}
\end{center}

\subsection*{Микро-задачи}
\begin{enumerate}
  \item Без да смяташ: коя е по-голяма за \(x\in(0,1)\) — \(x\) или \(x^2\)?
  \item Къде е \(x^2-4\) над оста \(x\) и къде е под нея?
  \item Как ще се промени графиката, ако добавим \(+3\): \(y=x^2+3\)?
\end{enumerate}

\subsection*{Отговори }
\begin{itemize}
  \item За \(x\in(0,1)\): \(x > x^2\).
  \item \(x^2-4\ge 0 \Leftrightarrow |x|\ge 2\): над оста за \(x\le -2\) или \(x\ge 2\), под оста за \(-2<x<2\).
  \item \(y=x^2+3\): цялата парабола се вдига нагоре с 3.
\end{itemize}

% =================================================
\subsection*{Още примери за квадратни функции}

\begin{center}
\begin{tikzpicture}
\begin{axis}[
    axis lines=middle,
    grid=both,
    xmin=-4,xmax=4,
    ymin=-6,ymax=10,
    xlabel={$x$}, ylabel={$y$},
    legend style={at={(0.02,0.98)},anchor=north west},
    width=11cm, height=7cm
]
\addplot[thick, blue, domain=-4:4] {x^2};
\addlegendentry{$y=x^2$}

\addplot[thick, red, domain=-4:4] {x^2-4};
\addlegendentry{$y=x^2-4$}

\addplot[thick, green!60!black, domain=-4:4] {(x-2)^2};
\addlegendentry{$y=(x-2)^2$}

\addplot[thick, purple, domain=-4:4] {-x^2+4};
\addlegendentry{$y=-x^2+4$}
\end{axis}
\end{tikzpicture}
\end{center}

% =================================================
\subsection*{Произволна квадратна функция}

Разглеждаме функцията:
\[
f(x)=x^2-2x-3
\]

\begin{center}
\begin{tikzpicture}
\begin{axis}[
    axis lines=middle,
    grid=both,
    xmin=-4,xmax=4,
    ymin=-6,ymax=6,
    xlabel={$x$}, ylabel={$y$},
    width=11cm, height=7cm
]
\addplot[thick, teal, domain=-4:4] {x^2-2*x-3};
\end{axis}
\end{tikzpicture}
\end{center}

% =================================================
\subsection*{Минимум и пресечни точки}

За функцията \(f(x)=x^2-2x-3\):

\begin{center}
\begin{tikzpicture}
\begin{axis}[
    axis lines=middle,
    grid=both,
    xmin=-4,xmax=4,
    ymin=-6,ymax=6,
    xlabel={$x$}, ylabel={$y$},
    width=11cm, height=7cm
]
% графика
\addplot[thick, blue, domain=-4:4] {x^2-2*x-3};

% връх (минимум)
\addplot[only marks, mark=*, mark size=2.5pt, red] coordinates {(1,-4)};
\node[red, above left] at (axis cs:1,-4) {минимум};

% ос на симетрия
\addplot[red!60, dashed] coordinates {(1,-6) (1,6)};
\node[red!60, below] at (axis cs:1,-6) {$x=1$};

% пресечни точки с x-оста
\addplot[only marks, mark=*, mark size=2.5pt, black] coordinates {(-1,0) (3,0)};
\node[below] at (axis cs:-1,0) {$-1$};
\node[below] at (axis cs:3,0) {$3$};

% пресечна точка с y-оста
\addplot[only marks, mark=*, mark size=2.5pt, black] coordinates {(0,-3)};
\node[left] at (axis cs:0,-3) {$-3$};

\end{axis}
\end{tikzpicture}
\end{center}

\subsection*{Кога имаме максимум?}

Ако коефициентът пред \(x^2\) е отрицателен (напр. \(g(x)=-x^2+4x+1\)),
параболата е отворена надолу и върхът е \textbf{максимум}.


% =================================================
\section*{5. Четни и нечетни функции (симетрии)}

\subsection*{Четна функция}
\[
f(-x)=f(x)
\]
Графика: симетрия спрямо оста \(y\). Примери: \(x^2\), \(\cos x\), \(|x|\).

\subsection*{Нечетна функция}
\[
f(-x)=-f(x)
\]
Графика: симетрия спрямо началото. Примери: \(x\), \(x^3\), \(\sin x\).

\subsection*{Графики: четна и нечетна}
\begin{center}
\begin{tikzpicture}
\begin{axis}[
    axis lines=middle, grid=both,
    xmin=-3,xmax=3,ymin=-2,ymax=9,
    xlabel={$x$}, ylabel={$y$},
    legend style={at={(0.02,0.98)},anchor=north west},
    width=11cm, height=7cm
]
\addplot[thick, purple, domain=-3:3] {x^2};
\addlegendentry{$y=x^2$ (четна)}
\addplot[thick, blue, domain=-3:3] {x^3/3};
\addlegendentry{$y=\frac{x^3}{3}$ (нечетна)}
\end{axis}
\end{tikzpicture}
\end{center}

\subsection*{Допълнителни графики (|x|, sin, cos)}
\begin{center}
\begin{tikzpicture}
\begin{axis}[
    axis lines=middle, grid=both,
    xmin=-6.5,xmax=6.5,ymin=-2.5,ymax=2.5,
    xlabel={$x$}, ylabel={$y$},
    legend style={at={(0.02,0.98)},anchor=north west},
    width=11cm, height=7cm
]
\addplot[thick, teal, domain=-6.28:6.28, samples=300] {sin(deg(x))};
\addlegendentry{$y=\sin x$ (нечетна)}
\addplot[thick, orange, domain=-6.28:6.28, samples=300] {cos(deg(x))};
\addlegendentry{$y=\cos x$ (четна)}
\addplot[thick, green!50!black, domain=-6.28:6.28, samples=300] {abs(x)/3};
\addlegendentry{$y=\frac{|x|}{3}$ (четна)}
\end{axis}
\end{tikzpicture}
\end{center}

\subsection*{Задачи (четност/нечетност)}
\begin{enumerate}
  \item Определи дали е четна/нечетна/нито: \(x^2+1\), \(x^3-x\), \(|x|\), \(\sin x+\cos x\).
  \item Ако \(f\) е четна, какво можеш да кажеш за \(f(5)\) и \(f(-5)\)?
  \item Ако \(g\) е нечетна, какво е \(g(0)\)?
\end{enumerate}

\subsection*{Отговори}
\begin{itemize}
  \item \(x^2+1\): четна. \ \ \(x^3-x\): нечетна. \ \ \(|x|\): четна. \ \ \(\sin x+\cos x\): нито.
  \item \(f(5)=f(-5)\).
  \item \(g(0)=0\).
\end{itemize}

% =================================================
\section*{6. Финитистка и инфинитистка математика}

\subsection*{Финитистки поглед}
\begin{itemize}
  \item работим само с крайни приближения
  \item безкрайността е „език“ за „много голямо“
\end{itemize}

\subsection*{Инфинитистки поглед}
\begin{itemize}
  \item безкрайността е легитимна идея
  \item границите могат да имат точни стойности
\end{itemize}


% =================================================
\section*{7. Раждането на границата чрез площ под \(x^2\) }

Искаме площта под графиката:
\[
y=x^2 \quad \text{за } x\in[0,1].
\]

\subsection*{Проблем}
Кривата е „гладка“, но не е правоъгълник или триъгълник.
Ние обаче умеем да намираме площ на правоъгълници.
Затова ще направим умишлено неточно приближение.

\subsection*{Стъпка 1: Разделяме интервала}
Разделяме \([0,1]\) на \(N\) равни части. Широчината на всяка е:
\[
\Delta x=\frac{1}{N}.
\]

\subsection*{Стъпка 2: Правоъгълници}
Ще построим правоъгълници при \(x=\frac{1}{N},\frac{2}{N},\dots,\frac{N-1}{N}\).

\begin{itemize}
  \item широчина: \(\frac{1}{N}\)
  \item височина: \(\left(\frac{1}{N}\right)^2,\left(\frac{2}{N}\right)^2,\dots,\left(\frac{N-1}{N}\right)^2\)
\end{itemize}

Лицата на правоъгълниците са:
\[
\frac{1}{N}\left(\frac{1}{N}\right)^2,\quad
\frac{1}{N}\left(\frac{2}{N}\right)^2,\quad
\dots,\quad
\frac{1}{N}\left(\frac{N-1}{N}\right)^2.
\]

\subsection*{Стъпка 3: Приближена площ }
Нека \(S_{\text{пр.}}\) е приближената площ (правоъгълници). Тогава:
\[
S_{\text{пр.}}
=
\frac{1}{N}\left(\frac{1}{N}\right)^2
+
\frac{1}{N}\left(\frac{2}{N}\right)^2
+
\cdots
+
\frac{1}{N}\left(\frac{N-1}{N}\right)^2.
\]

Сега изнасяме \(\frac{1}{N^3}\):
\[
S_{\text{пр.}}
=
\frac{1^2+2^2+\cdots+(N-1)^2}{N^3}.
\]

Това е \textbf{грешно} за крайно \(N\), но:

\begin{quote}
Колкото по-голямо е \(N\), толкова по-тесни са правоъгълниците и толкова по-малка е грешката.
\end{quote}

\subsection*{Стъпка 4: Идеята за граница}
Истинската площ е стойността, към която се доближава \(S_{\text{пр.}}\), когато \(N\to\infty\).
Ще използваме формулата за сбор на квадрати (която доказваме в края на материала):
\[
1^2+2^2+\cdots+m^2=\frac{m(m+1)(2m+1)}{6}.
\]

Тук \(m=N-1\). Следователно:
\[
1^2+2^2+\cdots+(N-1)^2=\frac{(N-1)N(2N-1)}{6}.
\]

Замествайки в \(S_{\text{пр.}}\):
\[
S_{\text{пр.}}
=
\frac{\frac{(N-1)N(2N-1)}{6}}{N^3}
=
\frac{(N-1)(2N-1)}{6N^2}.
\]

Разкриваме числителя:
\[
(N-1)(2N-1)=2N^2-3N+1.
\]

Тогава:
\[
S_{\text{пр.}}=\frac{2N^2-3N+1}{6N^2}
=
\frac{2}{6}-\frac{3}{6N}+\frac{1}{6N^2}.
\]

Когато \(N\to\infty\), последните два члена отиват към 0 и остава:
\[
\lim_{N\to\infty}S_{\text{пр.}}=\frac{1}{3}.
\]

% =================================================
\section*{8. Цветна графика със защрихована площ}

\begin{center}
\begin{tikzpicture}
\begin{axis}[
    axis lines=middle,
    xmin=0,xmax=1.1,
    ymin=0,ymax=1.1,
    grid=both,
    xlabel={$x$},
    ylabel={$y$},
    width=11cm,
    height=7cm
]
\addplot[domain=0:1, samples=200, thick, blue]{x^2};
\addplot[domain=0:1, samples=200, draw=none, fill=blue!25]{x^2}\closedcycle;
\end{axis}
\end{tikzpicture}
\end{center}

% =================================================
\section*{9. Правоъгълници}

\begin{center}
\begin{tikzpicture}
\begin{axis}[
    axis lines=middle,
    xmin=0,xmax=1.1,
    ymin=0,ymax=1.1,
    grid=both,
    xlabel={$x$},
    ylabel={$y$},
    width=11cm,
    height=7cm
]
% крива y = x^2
\addplot[domain=0:1, samples=200, thick, blue]{x^2};

% правоъгълници – долна сума (hardcoded)
\addplot[fill=red!30, draw=red] coordinates {
  (0.0, 0)   (0.0, 0.0)   (0.2, 0.0)   (0.2, 0)
};

\addplot[fill=red!30, draw=red] coordinates {
  (0.2, 0)   (0.2, 0.04)  (0.4, 0.04)  (0.4, 0)
};

\addplot[fill=red!30, draw=red] coordinates {
  (0.4, 0)   (0.4, 0.16)  (0.6, 0.16)  (0.6, 0)
};

\addplot[fill=red!30, draw=red] coordinates {
  (0.6, 0)   (0.6, 0.36)  (0.8, 0.36)  (0.8, 0)
};

\addplot[fill=red!30, draw=red] coordinates {
  (0.8, 0)   (0.8, 0.64)  (1.0, 0.64)  (1.0, 0)
};



\end{axis}
\end{tikzpicture}
\end{center}

\subsection*{Задачи }
\begin{enumerate}
  \item Ако взимаме височината в левия край, приближението ще е по-голямо или по-малко? (За \(x^2\) на \([0,1]\).)
  \item Какво ще стане с грешката, ако \(N\) стане 10, 100, 1000?
  \item Можем ли някога с крайно \(N\) да получим „перфектната“ площ за крива? Какво мислиш?
\end{enumerate}

\subsection*{Отговори }
\begin{itemize}
  \item За \(x^2\) на \([0,1]\) (растяща функция): ляв край дава \textbf{по-малко} приближение, десен край дава \textbf{по-голямо}.
  \item Грешката намалява, защото правоъгълниците стават по-тесни.
  \item С крайно \(N\) — почти никога „перфектно“ за крива; идеята на границата е да опишем какво става при безкрайно финно делене.
\end{itemize}

% =================================================
\section*{10. Граници: задачи 1–17 }

\subsection*{Полезни техники}
\begin{itemize}
  \item При \(x\to\infty\): сравняваме най-високите степени или делим на най-високата степен.
  \item Факторизация: \(x^2-a^2=(x-a)(x+a)\).
  \item Рационализация: умножаваме по „спрегнатото“ (същото, но със сменен знак).
  \item За корени от степен 3: \(a^3-b^3=(a-b)(a^2+ab+b^2)\).
\end{itemize}

% ------------------ 1 ------------------
\subsection*{1. \(\displaystyle \lim_{x\to\infty}\frac{x^2-3x+4}{x+4}\)}

\textbf{Решение:}  
Когато \(x\) е много голямо, най-важни са членовете с най-висока степен.
В числителя имаме \(x^2\), а в знаменателя имаме \(x\). Това подсказва, че дробта расте приблизително като \(x\).

Най-сигурният начин: делим числителя на знаменателя (или преобразуваме):

\[
\frac{x^2-3x+4}{x+4}
=
\frac{(x+4)(x-7)+32}{x+4}
\]
защото:
\[
(x+4)(x-7)=x^2-7x+4x-28=x^2-3x-28.
\]
За да стане \(x^2-3x+4\), добавяме 32.

Тогава:
\[
\frac{x^2-3x+4}{x+4} = x-7 + \frac{32}{x+4}.
\]

Сега гледаме границата:
\[
\lim_{x\to\infty}\left(x-7+\frac{32}{x+4}\right).
\]
Членът \(\frac{32}{x+4}\to 0\), но \(x-7\to +\infty\). Следователно:
\[
\boxed{+\infty}.
\]

% ------------------ 2 ------------------
\subsection*{2. \(\displaystyle \lim_{x\to\infty}\left(\frac{2x-1}{3x+1}-\frac{2x^2+3}{3x^2-1}\right)\)}

\textbf{Решение:}  
Това е разлика на две дроби, и за \(x\to\infty\) можем да намерим границата на всяка отделно.

\textbf{Първа дроб:}
\[
\frac{2x-1}{3x+1}.
\]
Делим числител и знаменател на \(x\):
\[
\frac{2-\frac{1}{x}}{3+\frac{1}{x}} \xrightarrow[x\to\infty]{} \frac{2-0}{3+0}=\frac{2}{3}.
\]

\textbf{Втора дроб:}
\[
\frac{2x^2+3}{3x^2-1}.
\]
Делим числител и знаменател на \(x^2\):
\[
\frac{2+\frac{3}{x^2}}{3-\frac{1}{x^2}} \xrightarrow[x\to\infty]{} \frac{2+0}{3-0}=\frac{2}{3}.
\]

Значи цялата граница е:
\[
\frac{2}{3}-\frac{2}{3}=0
\]
\[
\boxed{0}.
\]

% ------------------ 3 ------------------
\subsection*{3. \(\displaystyle \lim_{x\to\infty}\frac{x^2-4}{x-2}\)}

\textbf{Решение:}  
Факторизираме разлика на квадрати:
\[
x^2-4=(x-2)(x+2).
\]
Тогава за \(x\ne 2\):
\[
\frac{x^2-4}{x-2}=\frac{(x-2)(x+2)}{x-2}=x+2.
\]
Сега:
\[
\lim_{x\to\infty}(x+2)=+\infty.
\]
\[
\boxed{+\infty}.
\]

% ------------------ 4 ------------------
\subsection*{4. \(\displaystyle \lim_{x\to\infty}\frac{x^4+3x^2}{x^5+x^3+2x^2}\)}

\textbf{Решение:}  
Най-високата степен в знаменателя е \(x^5\). Делим всичко на \(x^5\):
\[
\frac{\frac{x^4}{x^5}+\frac{3x^2}{x^5}}{\frac{x^5}{x^5}+\frac{x^3}{x^5}+\frac{2x^2}{x^5}}
=
\frac{\frac{1}{x}+\frac{3}{x^3}}{1+\frac{1}{x^2}+\frac{2}{x^3}}.
\]
При \(x\to\infty\) всички \(\frac{1}{x},\frac{1}{x^2},\frac{1}{x^3}\to 0\). Значи:
\[
\frac{0+0}{1+0+0}=0
\]
\[
\boxed{0}.
\]

% ------------------ 5 ------------------
\subsection*{5. \(\displaystyle \lim_{x\to\infty}\frac{x^2-5x+4}{x^2+x-2}\)}

\textbf{Решение:}  
Най-високата степен е \(x^2\). Делим на \(x^2\):
\[
\frac{1-\frac{5}{x}+\frac{4}{x^2}}{1+\frac{1}{x}-\frac{2}{x^2}}
\xrightarrow[x\to\infty]{}
\frac{1-0+0}{1+0-0}=1.
\]
\[
\boxed{1}.
\]

% ------------------ 6 ------------------
\subsection*{6. \(\displaystyle \lim_{x\to\infty}\left(\frac{3x}{5x-1}\cdot\frac{x^2+1}{x^2+2x-1}\right)\)}

\textbf{Решение:}  
Имаме произведение на две дроби. Намираме границите им поотделно.

\textbf{Първа дроб:}
\[
\frac{3x}{5x-1}.
\]
Делим на \(x\):
\[
\frac{3}{5-\frac{1}{x}} \xrightarrow[x\to\infty]{} \frac{3}{5}.
\]

\textbf{Втора дроб:}
\[
\frac{x^2+1}{x^2+2x-1}.
\]
Делим на \(x^2\):
\[
\frac{1+\frac{1}{x^2}}{1+\frac{2}{x}-\frac{1}{x^2}}
\xrightarrow[x\to\infty]{}
\frac{1+0}{1+0-0}=1.
\]

Произведението:
\[
\frac{3}{5}\cdot 1=\frac{3}{5}.
\]
\[
\boxed{\frac{3}{5}}.
\]

% ------------------ 7 ------------------
\subsection*{7. \(\displaystyle \lim_{x\to\infty}\frac{ax^3+b}{cx^3+d}\quad (c\neq 0)\)}

\textbf{Решение:}  
Най-високата степен е \(x^3\). Делим числител и знаменател на \(x^3\):
\[
\frac{a+\frac{b}{x^3}}{c+\frac{d}{x^3}}.
\]
При \(x\to\infty\): \(\frac{b}{x^3}\to 0\) и \(\frac{d}{x^3}\to 0\). Получаваме:
\[
\frac{a}{c}.
\]
\[
\boxed{\frac{a}{c}}.
\]

% ------------------ 8 ------------------
\subsection*{8. \(\displaystyle \lim_{x\to\infty}\frac{-3-x}{x^2+2x+3}\)}

\textbf{Решение:}  
Степента на числителя е 1, на знаменателя е 2. Дробта се държи като \(\frac{x}{x^2}=\frac{1}{x}\to 0\).
Да го направим формално: делим на \(x^2\):
\[
\frac{\frac{-x}{x^2}+\frac{-3}{x^2}}{1+\frac{2}{x}+\frac{3}{x^2}}
=
\frac{-\frac{1}{x}-\frac{3}{x^2}}{1+\frac{2}{x}+\frac{3}{x^2}}
\xrightarrow[x\to\infty]{}
\frac{0-0}{1+0+0}=0.
\]
\[
\boxed{0}.
\]

% ------------------ 9 ------------------
\subsection*{9. \(\displaystyle \lim_{x\to\infty}\frac{(x+3)(x+5)}{x^2+1}\)}

\textbf{Решение:}  
Разкриваме числителя:
\[
(x+3)(x+5)=x^2+8x+15.
\]
Тогава:
\[
\frac{x^2+8x+15}{x^2+1}.
\]
Делим на \(x^2\):
\[
\frac{1+\frac{8}{x}+\frac{15}{x^2}}{1+\frac{1}{x^2}}
\xrightarrow[x\to\infty]{}
\frac{1+0+0}{1+0}=1.
\]
\[
\boxed{1}.
\]

% ------------------ 10 ------------------
\subsection*{10. \(\displaystyle \lim_{x\to 0}\frac{\sqrt{x+1}-1}{x}\)}

\textbf{Решение:}  
Тук директното заместване дава \(0/0\). Трикът е рационализация.

Умножаваме числителя и знаменателя по спрегнатото \(\sqrt{1+x}+1\):
\[
\frac{\sqrt{1+x}-1}{x}\cdot \frac{\sqrt{1+x}+1}{\sqrt{1+x}+1}
=
\frac{(1+x)-1}{x(\sqrt{1+x}+1)}
=
\frac{x}{x(\sqrt{1+x}+1)}
=
\frac{1}{\sqrt{1+x}+1}.
\]
Сега вече можем да заместим \(x=0\):
\[
\frac{1}{\sqrt{1}+1}=\frac{1}{2}.
\]
\[
\boxed{\frac{1}{2}}.
\]

% ------------------ 11 ------------------
\subsection*{11. \(\displaystyle \lim_{x\to 0}\frac{\sqrt{1+x}-\sqrt{1-x}}{4x}\)}

\textbf{Решение:}  
Пак имаме \(0/0\). Рационализираме числителя чрез умножение по спрегнатото:
\[
\frac{\sqrt{1+x}-\sqrt{1-x}}{4x}\cdot
\frac{\sqrt{1+x}+\sqrt{1-x}}{\sqrt{1+x}+\sqrt{1-x}}
=
\frac{(1+x)-(1-x)}{4x\left(\sqrt{1+x}+\sqrt{1-x}\right)}.
\]
Числителят:
\[
(1+x)-(1-x)=2x.
\]
Тогава:
\[
\frac{2x}{4x(\sqrt{1+x}+\sqrt{1-x})}
=
\frac{1}{2(\sqrt{1+x}+\sqrt{1-x})}.
\]
Сега заместваме \(x=0\):
\[
\frac{1}{2(1+1)}=\frac{1}{4}.
\]
\[
\boxed{\frac{1}{4}}.
\]

% ------------------ 12 ------------------
\subsection*{12. \(\displaystyle \lim_{x\to 0}\frac{\sqrt[3]{1+x^2}-1}{x}\)}

\textbf{Решение:}  
Директно е \(0/0\). Тук рационализацията е чрез формулата за разлика на кубове.

Нека:
\[
a=\sqrt[3]{1+x^2},\qquad b=1.
\]
Тогава:
\[
a^3-b^3=(a-b)(a^2+ab+b^2).
\]
Но:
\[
a^3-b^3=(1+x^2)-1=x^2.
\]
Следователно:
\[
a-b=\frac{x^2}{a^2+ab+b^2}.
\]
Това значи:
\[
\sqrt[3]{1+x^2}-1=\frac{x^2}{a^2+a+1}.
\]
Делим на \(x\):
\[
\frac{\sqrt[3]{1+x^2}-1}{x}
=
\frac{x^2}{x(a^2+a+1)}
=
\frac{x}{a^2+a+1}.
\]
Когато \(x\to 0\), имаме \(a=\sqrt[3]{1+x^2}\to 1\), така че знаменателят \(\to 1^2+1+1=3\),
а числителят \(x\to 0\). Значи:
\[
\boxed{0}.
\]

% ------------------ 13 ------------------
\subsection*{13. \(\displaystyle \lim_{x\to 5}\frac{\sqrt{x-1}-2}{x-5}\)}

\textbf{Решение:}  
Директно е \(0/0\). Рационализираме числителя:
\[
\frac{\sqrt{x-1}-2}{x-5}\cdot\frac{\sqrt{x-1}+2}{\sqrt{x-1}+2}
=
\frac{(x-1)-4}{(x-5)(\sqrt{x-1}+2)}
=
\frac{x-5}{(x-5)(\sqrt{x-1}+2)}
=
\frac{1}{\sqrt{x-1}+2}.
\]
Заместваме \(x=5\):
\[
\frac{1}{\sqrt{4}+2}=\frac{1}{4}.
\]
\[
\boxed{\frac{1}{4}}.
\]

% ------------------ 14 ------------------
\subsection*{14. \(\displaystyle \lim_{x\to -3}\frac{x+3}{\sqrt{x+4}-1}\)}

\textbf{Решение:}  
Директно е \(0/0\). Рационализираме знаменателя:
\[
\frac{x+3}{\sqrt{x+4}-1}\cdot\frac{\sqrt{x+4}+1}{\sqrt{x+4}+1}
=
\frac{(x+3)(\sqrt{x+4}+1)}{(x+4)-1}
=
\frac{(x+3)(\sqrt{x+4}+1)}{x+3}.
\]
За \(x\neq -3\) съкращаваме \(x+3\):
\[
\sqrt{x+4}+1.
\]
Заместваме \(x=-3\):
\[
\sqrt{1}+1=2.
\]
\[
\boxed{2}.
\]

% ------------------ 15 ------------------
\subsection*{15. \(\displaystyle \lim_{x\to 0}\frac{\sqrt{x+n}-\sqrt{n}}{x}\quad (n>0)\)}

\textbf{Решение:}  
Директно е \(0/0\). Рационализираме:
\[
\frac{\sqrt{n+x}-\sqrt{n}}{x}\cdot\frac{\sqrt{n+x}+\sqrt{n}}{\sqrt{n+x}+\sqrt{n}}
=
\frac{(n+x)-n}{x(\sqrt{n+x}+\sqrt{n})}
=
\frac{x}{x(\sqrt{n+x}+\sqrt{n})}
=
\frac{1}{\sqrt{n+x}+\sqrt{n}}.
\]
Сега \(x\to 0\):
\[
\frac{1}{\sqrt{n}+\sqrt{n}}=\frac{1}{2\sqrt{n}}.
\]
\[
\boxed{\frac{1}{2\sqrt{n}}}.
\]

% ------------------ 16 ------------------
\subsection*{16. \(\displaystyle \lim_{x\to 16}\frac{\sqrt[4]{x}-2}{\sqrt{x}-4}\)}

\textbf{Решение:}  
Директно е \(0/0\). Тук най-чисто е замяна.

Нека:
\[
t=\sqrt[4]{x}.
\]
Тогава:
\[
\sqrt{x}=x^{1/2}=(x^{1/4})^2=t^2.
\]
Когато \(x\to 16\), \(\sqrt[4]{16}=2\), тоест \(t\to 2\).

Изразът става:
\[
\frac{t-2}{t^2-4}.
\]
Факторизираме:
\[
t^2-4=(t-2)(t+2).
\]
Съкращаваме \(t-2\) (за \(t\neq 2\)):
\[
\frac{1}{t+2}.
\]
Сега заместваме \(t=2\):
\[
\frac{1}{4}.
\]
\[
\boxed{\frac{1}{4}}.
\]

% ------------------ 17 ------------------
\subsection*{17. \(\displaystyle \lim_{x\to 1}\frac{\sqrt{x+3}-2}{x-1}\)}

\textbf{Решение:}  
Директно е \(0/0\). Рационализираме числителя:
\[
\frac{\sqrt{x+3}-2}{x-1}\cdot\frac{\sqrt{x+3}+2}{\sqrt{x+3}+2}
=
\frac{(x+3)-4}{(x-1)(\sqrt{x+3}+2)}
=
\frac{x-1}{(x-1)(\sqrt{x+3}+2)}
=
\frac{1}{\sqrt{x+3}+2}.
\]
Заместваме \(x=1\):
\[
\frac{1}{\sqrt{4}+2}=\frac{1}{4}.
\]
\[
\boxed{\frac{1}{4}}.
\]

% =================================================
\section*{11. Доказателство за формулата за сбор на квадрати}

Цел:
\[
1^2+2^2+3^2+\cdots+n^2=\frac{n(n+1)(2n+1)}{6}.
\]

\subsection*{Стъпка 1: Разлика на кубове за два последователни члена}

Започваме от тъждеството за разлика на кубове:
\[
a^3-b^3=(a-b)\left(a^2+ab+b^2\right).
\]

Избираме \(a=n\) и \(b=n-1\). Тогава \(a-b=1\), и получаваме:
\[
n^3-(n-1)^3=(n-(n-1))\Big(n^2+n(n-1)+(n-1)^2\Big)
=1\cdot\Big(n^2+n(n-1)+(n-1)^2\Big).
\]

Сега разкриваме скобата:
\[
n^2+n(n-1)+(n-1)^2
=n^2+(n^2-n)+(n^2-2n+1)
=3n^2-3n+1.
\]

Следователно:
\[
\boxed{\,n^3-(n-1)^3=3n^2-3n+1\,}.
\]

\subsection*{Стъпка 2: Пишем същото за всички стъпки надолу}

По същия начин:
\[
(n-1)^3-(n-2)^3=3(n-1)^2-3(n-1)+1,
\]
\[
(n-2)^3-(n-3)^3=3(n-2)^2-3(n-2)+1,
\]
\[
\vdots
\]
\[
2^3-1^3=3\cdot 2^2-3\cdot 2+1,
\]
\[
1^3-0^3=3\cdot 1^2-3\cdot 1+1.
\]

\subsection*{Стъпка 3: Събираме всички равенства}

Събираме лявата страна на всички равенства:
\[
\big(n^3-(n-1)^3\big)+\big((n-1)^3-(n-2)^3\big)+\cdots+\big(1^3-0^3\big).
\]

Всичко по средата се съкращава:
\[
-(n-1)^3+(n-1)^3,\quad -(n-2)^3+(n-2)^3,\ \ldots,\ -1^3+1^3,
\]
и остава само:
\[
n^3-0^3=n^3.
\]

Сега събираме дясната страна. Получаваме:
\[
n^3
=
\big(3n^2-3n+1\big)+\big(3(n-1)^2-3(n-1)+1\big)+\cdots+\big(3\cdot 1^2-3\cdot 1+1\big).
\]

Групираме „по типове“ членове:

\begin{itemize}
  \item квадратите: \(3\big(1^2+2^2+\cdots+n^2\big)\)
  \item линейните: \(-3\big(1+2+\cdots+n\big)\)
  \item единиците: \(1+1+\cdots+1\) (общо \(n\) пъти) \(\Rightarrow n\)
\end{itemize}

Тоест:
\[
\boxed{\,
n^3 = 3\sum_{k=1}^{n}k^2 \;-\; 3\sum_{k=1}^{n}k \;+\; n
\,}.
\]

\subsection*{Стъпка 4: Изолираме \(\sum k^2\)}

Преместваме останалите членове:
\[
3\sum_{k=1}^{n}k^2
=
n^3 + 3\sum_{k=1}^{n}k - n.
\]

Делим на 3:
\[
\sum_{k=1}^{n}k^2
=
\frac{1}{3}\Big(n^3 + 3\sum_{k=1}^{n}k - n\Big).
\]

Знаем формулата за сумата на първите \(n\) естествени числа:
\[
\sum_{k=1}^{n}k=\frac{n(n+1)}{2}.
\]

Замествайки:
\[
\sum_{k=1}^{n}k^2
=
\frac{1}{3}\left(n^3 + 3\cdot\frac{n(n+1)}{2} - n\right).
\]

\subsection*{Стъпка 5: Опростяване до класическата формула}

Изнасяме \(n\):
\[
\sum_{k=1}^{n}k^2
=
\frac{n}{3}\left(n^2 + \frac{3(n+1)}{2} - 1\right).
\]

Събираме в една дроб:
\[
n^2 + \frac{3(n+1)}{2} - 1
=
\frac{2n^2 + 3(n+1) - 2}{2}
=
\frac{2n^2+3n+1}{2}.
\]

Тогава:
\[
\sum_{k=1}^{n}k^2
=
\frac{n}{3}\cdot \frac{2n^2+3n+1}{2}
=
\frac{n(2n^2+3n+1)}{6}.
\]

Факторизираме:
\[
2n^2+3n+1=(2n+1)(n+1).
\]

Накрая получаваме:
\[
\boxed{\sum_{k=1}^{n}k^2=\frac{n(n+1)(2n+1)}{6}}.
\]



\end{document}
